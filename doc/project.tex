\documentclass{article}

\usepackage{verbatim}
\usepackage{xspace}

\newcommand{\fct}{FC\(_2\)\xspace}
\newcommand{\kindc}{\(C\)\xspace}
\newcommand{\kindt}{\(T\)\xspace}

\title{CIS 670 --- Final project}
\author{Arthur Azevedo de Amorim \\ Hongbo Zhang}

\begin{document}

\maketitle

\section{Introduction}

With \emph{generative type abstraction}, the implementation of a
module can define a new type that is a synonym of some existing type
while hiding this representation from clients of that module. Thus,
whereas the implementation can treat the new type as an alias of the
original type, code that uses that module is forced to use the
exported interface, which might not even expose the actual
representation of that type.

A problem arises when trying to combine this feature with other
non-parametric features, such as type functions. Indeed, in some
cases, it might be desirable for a type function to yield different
results for two types even though those types are synonymous. Hence,
in order to add these two features to a language in a sound manner, it
is necessary to limit the extent to which two types can be synonymous.

In~\cite{newtypes}, the authors extend System FC, the core language of
the Glasgow Haskell Compiler, with kind annotations that allow one to
distinguish between parametric and non-parametric contexts in a
program. Thus, when two types are declared synonymous, they can only
be used as such in some contexts, while in others they are treated as
being different.

In our work, we've implemented a slightly modified version of the \fct
type checker. Section~\ref{sec:description} discusses the main
differences in the type system of \fct compared to that of System
FC. Section~\ref{sec:implementation} describes our implementation of
the type checker.

\section{Description}

\label{sec:description}

\fct adds two \emph{kind roles} to base System FC, noted \kindc (for
``code'') and \kindt (for ``type''). These mark two different kinds of
equivalence and annotate type arguments in kinds. Equivalence at role
\kindc is a finer notion than equivalence at role \kindt: types that
are equivalent at role \kindc are equivalent at role \kindt, but the
converse does not necessarily hold.

Intuitively, type functions that take arguments at role \kindt care
only about the representation of the types, but not about their
names. Thus, if types \texttt{Int} and \texttt{Age} are declared to be
synonymous at role \kindt, types \texttt{List Int} and \texttt{List
  Age} should be equivalent, even though they are symbolically
different. This is the intended behavior for regular Haskell algebraic
data types. However, if a type function takes arguments at role
\kindc, it'll distinguish types that have the same representation but
different names. This case corresponds to indexed type families: it is
possible to define a type family \texttt{F} such that \texttt{F Int =
  Char} but \texttt{F Age = Bool}. Thus, it would be an error to
consider \texttt{F Int} and \texttt{F Age} as synonymous, even though
the two arguments have the same representation.

As it stands, the \fct type system and context formation rules are too
permisive to guarantee basic important properties, such as
preservation and progress. Indeed, it is possible to add inconsistent
type equivalence assumptions in the context, such as declaring that
two distinct data types, such as \texttt{List a} and \texttt{Maybe a},
are equivalent. Hence, it is necessary to assume that a context is
\emph{consistent} in order to prove these safety theorems. The authors
define context consistency and a sufficient property to show that a
context is consistent. They also prove preservation and progress in
these conditions for \fct.

Finally, the authors show how to translate standard Haskell type
declarations into a suitable \fct environment. The translation is such
that appropriate roles are added to kinds in order to ensure that
parametric and non-parametric contexts are treated correctly.

\section{Implementation and Discussion}

\label{sec:implementation}

We've implemented a type checker for a version of \fct in OCaml. Our
program follows essentially the typing rules shown in the paper. The
rules are syntax-directed, so the translation was relatively
straight-forward.

While working with the original version \fct, we found that it would
be worth adding some slight modifications to make the system more
convenient to work with. The main difference lays in the way the
context of a program is described. Instead of describing each part of
the context separately, as done in the original system, our type
declarations are modeled after Haskell's. Hence, a generic list data
type is declared as
\begin{verbatim}
DATA List (a:*/T) WHERE {
  Cons :: {(a:*/T)} (a -> ((List a) -> (List a)))
  Nil :: {(a:*/T)} (List a)
};
\end{verbatim}
while a Haskell-like newtype could be declared as
\begin{verbatim}
NEWTYPE Age = MkAge Nat;
\end{verbatim}

Notice the explicit kind annotations: instead of inferring the best
kinds and roles for each declaration, we decided to stick to the
original system in most of the cases. Internally, the environment is
still represented as a list of bindings. The notation is translated to
the original form during parse time, inspired by the compilation rules
presented in the paper. Our parser was written using ocamllex and
ocamlyacc.

We've also added the possibility of binding terms to names in the
language using an environment for terms.

The type checker is divided into three components: a type checker for
terms, which is the main function, a coercion proof checker and a kind
checker for types. Implementing most of the cases was
straight-forward: the most difficult one was typing a case statement,
which requires manipulating the environment doing several
verifications, but it was still fairly simple.

Due to time concerns, our implementation lacks some important
features. It would be really useful to add some sanity checks to the
environment, such as ensuring that it is well-formed or that the
coercion assumptions are consistent, following the method presented in
the paper. We also haven't implemented an evaluator for the
system. Adding these features would be interesting directions for
future work.

As a side note, while implementing, one of the main difficulties was
trying to reason about the system while drawing our intuition from
Haskell. Indeed, many features in Haskell, such as GADT's and
existential types, are not directly present in System FC or \fct and
need to be translated to those languages, sometimes in non-obvious ways.

Designing an explicit concrete syntax of our language is an
engineering work, however,it takes some time to make a robust
parser. We did just one pass for the simple language, type-check was
done during the parse time.
\section{Conclusion}

We've implemented a type checker for \fct, an extension of the core
language of GHC that solves some problems related to type coercions in
parametric and non-parametric contexts. The implementation is 
basic, but captures the essence of its type system.

\bibliography{project}{}
\bibliographystyle{plain}

\end{document}
